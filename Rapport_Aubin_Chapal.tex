%
% file : Rapport_Aubin_Chapal.tex
% date : lun. nov. 12 17:36:23 CET 2018
% author : vchapal
% description :
%
\documentclass{article}
\usepackage[utf8]{inputenc}
\usepackage{amsmath, amssymb}

\begin{document}

\title{Rapport Devoir Maison - Jeu Light Up}
\author{Claire AUBIN et Victor CHAPAL}
\maketitle

\section{Jeu sans les murs}
Dans cette section nous ne traiterons que les contraintes 1, 2 et 4. En effet les autres ne nous concernent pas pour le moment car elles dépendent des murs.

\begin{itemize}
\item Contrainte 1 : $ \wedge_{i,j \in [1,N]} ~ islit_{i,j}  $ \\
 Nous considérons qu'une case est vide si elle ne contient pas de mur. Donc pour cette partie toutes les cases sont vides.
  
\item Contrainte 2 : $ islit_{i,j} ~ \Leftrightarrow ~ bulb_{i,j} ~ \vee ~ (\vee_{i \in [1,N]} ~ bulb_{i,j}) ~ \vee ~ (\vee_{j \in [1,N]} ~ bulb_{i,j}) $\\
  La case (i,j) est éclairée si elle contient une lampe ou s'il y a une lampe sur la ligne i ou sur la colonne j.
  
\item Contrainte 4 : $ bulb_{i,j} ~ \Rightarrow ~ (\wedge_{i \in [1,N]} ~ \neg bulb_{i,j}) ~ \wedge ~ (\wedge_{j \in [1,N]} ~ \neg bulb_{i,j})  $ \\
  Si une lampe se trouve sur la case (i,j), il ne peut pas y avoir de lampe ni sur la ligne i, ni sur la colonne j.
\end{itemize}




\section{Ajout des murs sans cardinalit\'e}

Tout d'abord nous commençons par d\'efinir les deux formules suivnates :
\begin{itemize}
\item $ nowall_{(i,j)(k,l)} ~ \Leftrightarrow ~ \wedge_{m,n \in [i,k]\times[j,l]} ~ \neg wall_{m,n} $
\item $ haswall_{(i,j)(k,l)} ~ \Leftrightarrow ~ \neg nowall_{(i,j)(k,l)} $\\
\end{itemize}
Ce qui nous donne les contraintes suivantes :
\begin{itemize}
  
\item Contrainte 1 : $ nowall_{(i,j)(k,l)} ~ \Rightarrow ~ \wedge_{m,n \in [i,k]\times[j,l]} ~ islit_{m,n} $ \\
  Ici nous redéfinissons cette contrainte car elle n'est plus valable avec l'ajout des murs. 
\item Contrainte 3 :
\begin{align}
\nonumber
 wall_{i,j} ~ \Rightarrow &~ [(\vee_{m<i}bulb_{m,j} ~ \Rightarrow ~ \wedge_{m>i} ~ \neg islit_{m,j})~\wedge ~ (\wedge_{m>i} ~ bulb_{m,j} ~ \Rightarrow ~ \wedge_{m<i} ~ \neg islit_{m,j})\\
\nonumber
& \wedge ~ (\wedge_{n<j} ~ bulb_{i,n} ~ \Rightarrow ~ \wedge_{n>j} ~ \neg islit_{i,n})~ \wedge ~ (\wedge_{n>j} ~ bulb_{i,n} ~ \Rightarrow ~ \wedge_{n<j} ~ \neg islit_{i,n})]\\
\nonumber
\end{align}
Si un mur se trouve sur la case (i,j), alors une lampe qui serait située au nord du mur, ne pourrais pas propoger sa lumière au sur du mur et inverssement. De même si une lampe se trouve à droite du mur, elle ne peut pas propager sa lumière à gauche du mur et inverssement.
\item Contrainte 5 : $ wall_{i,j} ~ \Rightarrow ~ \neg bulb_{i,j} $\\
S'il y a un mur sur la case (i,j), il ne peut pas y avoir de lampe.
\end{itemize}
Nous ne traitons pas les contraintes 2 et 4 dans cette partie. En effet ces contraintes ne font pas directement réf\'erences aux murs donc ces dernière n'ont pas besoin d'être modifiés. 

\section{Ajout des cardinalit\'es}

Tout d'abord nous commen\c{c}ons par d\'efinir les formules suivantes :
\begin{itemize}
\item $ card_{(i,j),0} ~ \Leftrightarrow ~ \neg bulb_{i,j} $
\item $ card_{(i,j),1} ~ \Leftrightarrow ~ bulb_{i,j} $
\item 
\begin{align}
\nonumber card(i,j),n,e,s,w ~ \Leftrightarrow &~card(i-1,j),n ~ \wedge ~ card(i,j+1),e
\\ 
\nonumber
&~ \wedge ~ card(i+1,j),s ~ \wedge ~ card(i,j-1),w 
\end{align}
\end{itemize}
Cela nous permet ainsi de r\'epondre \`a la derni\`ere contrainte :
\begin{align}
\nonumber
&card0(i,j) ~ \Leftrightarrow ~ card(i-1,j),0 ~ \wedge ~ card(i,j+1),0~ \wedge ~ card(i+1,j),0 ~ \wedge ~ card(i,j-1),0\\
\nonumber
\\
\nonumber
&card1(i,j) ~ \Leftrightarrow ~ \vee_{n+e+s+w = 1} ~ [card(i-1,j),n ~ \wedge ~ card(i,j+1),e ~ \wedge ~ card(i+1,j),s ~ \wedge ~ card(i,j-1),w]\\
\nonumber
\\
\nonumber
&card2(i,j) ~ \Leftrightarrow ~ \vee_{n+e+s+w = 2} ~ [card(i-1,j),n ~ \wedge ~ card(i,j+1),e~ \wedge ~ card(i+1,j),s ~ \wedge ~ card(i,j-1),w]\\
\nonumber
\\
\nonumber
&card3(i,j) ~ \Leftrightarrow ~ \vee_{n+e+s+w = 3} ~ [card(i-1,j),n ~ \wedge ~ card(i,j+1),e~ \wedge ~ card(i+1,j),s ~ \wedge ~ card(i,j-1),w]\\
\nonumber
\\
\nonumber
&card4(i,j) ~ \Leftrightarrow ~ card(i-1,j),1 ~ \wedge ~ card(i,j+1),1 ~ \wedge ~ card(i+1,j),1 ~ \wedge ~ card(i,j-1),1\\
\nonumber
\end{align}

\end{document}
