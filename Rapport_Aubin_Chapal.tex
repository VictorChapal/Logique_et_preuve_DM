%
% file : Rapport_Aubin_Chapal.tex
% date : lun. nov. 12 17:36:23 CET 2018
% author : vchapal
% description :
%
\documentclass{article}
\usepackage{amsmath, amssymb}

\begin{document}

\title{Rapport Devoir Maison - Jeu Light Up}
\author{Claire AUBIN et Victor CHAPAL}
\maketitle

\section{Jeu sans les murs}
Dans cette section nous ne traiterons que les contraintes 1, 2 et 4. En effet les autres ne nous concernent pas pour le moment.

\begin{itemize}
\item Contrainte 1 : $ \wedge_{i,j \in [1,N]} ~ islit_{i,j}  $
\item Contrainte 2 : $ islit_{i,j} ~ \Leftrightarrow ~ bulb_{i,j} ~ \vee ~ (\vee_{i \in [1,N]} ~ bulb_{i,j}) ~ \vee ~ (\vee_{j \in [1,N]} ~ bulb_{i,j}) $
\item Contrainte 4 : $ bulb_{i,j} ~ \Rightarrow ~ (\wedge_{i \in [1,N]} ~ \neg bulb_{i,j}) ~ \wedge ~ (\wedge_{j \in [1,N]} ~ \neg bulb_{i,j})  $
\end{itemize}

\section{Ajout des murs sans cardinalit\'e}

Tout d'abord nous commençons par définir les deux formules suivnates :
\begin{itemize}
\item $ nowall_{(i,j)(k,l)} ~ \Leftrightarrow ~ \wedge_{m,n \in [i,k]\times[j,l]} ~ \neg wall_{m,n} $
\item $ haswall_{(i,j)(k,l)} ~ \Leftrightarrow ~ \neg nowall_{(i,j)(k,l)} $\\
\end{itemize}
Ce qui nous donne les contraintes suivantes :
\begin{itemize}
\item Contrainte 1 : $ nowall_{(i,j)(k,l)} ~ \Rightarrow ~ \wedge_{m,n \in [i,k]\times[j,l]} ~ islit_{i,j} $
\item Contrainte 2 : Identique \`a la formule de la partie sans les murs. En effet cette contrainte ne fait pas directement ref\'erence aux murs.
\item Contrainte 3 :
\begin{align}
\nonumber
 wall_{i,j} ~ \Rightarrow &~ [(\vee_{m<i}bulb_{m,j} ~ \Rightarrow ~ \wedge_{m>i} ~ \neg islit_{m,j})~\wedge ~ (\wedge_{m>i} ~ bulb_{m,j} ~ \Rightarrow ~ \wedge_{m<i} ~ \neg islit_{m,j})\\
\nonumber
& \wedge ~ (\wedge_{n<j} ~ bulb_{i,n} ~ \Rightarrow ~ \wedge_{n>j} ~ \neg islit_{i,n})~ \wedge ~ (\wedge_{n>j} ~ bulb_{i,n} ~ \Rightarrow ~ \wedge_{n<j} ~ \neg islit_{i,n})]\\
\nonumber
\end{align}
\item Contrainte 4 : Identique \'a la formule de la partie sans les murs. En effet cette contrainte ne fait pas directement ref\'erence aux murs.
\item Contrainte 5 : $ wall_{i,j}~\Rightarrow~\neg bulb_{i,j} $
\end{itemize}

\section{Ajout des cardinalités}

Tout d'abord nous commençons par définir les deux formules suivnates :
\begin{itemize}
\item $ card_{(i,j),0}~\Leftrightarrow~\neg bulb_{i,j} $
\item $ card_{(i,j),1}~\Leftrightarrow~bulb_{i,j} $\\
\end{itemize}
Ce qui nous donne les contraintes suivantes :
\begin{itemize}
\item Contrainte 1 :
\item Contrainte 2 :
\item Contrainte 3 :
\item Contrainte 4 :
\item Contrainte 5 :
\item Contrainte 6 :
\end{itemize}

\end{document}
